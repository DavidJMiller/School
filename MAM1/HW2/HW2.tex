\documentclass[12pt]{article}

%-------------PACKAGES------------- 
\usepackage[margin=1in]{geometry} 
\usepackage{amsmath,amsthm,amssymb}
\usepackage{pgfplots}
\usepackage{float}
\usepackage{braket}
\usepackage{titling}
\usepackage{tikz}

%-------------FORMATTING-------------
\setlength{\droptitle}{-5em} 
\setlength{\parindent}{0pt}
 
%--------------COMMANDS--------------
\newcommand{\N}{\mathbb{N}}
\newcommand{\Z}{\mathbb{Z}}
\newcommand{\R}{\mathbb{R}}
\newcommand{\C}{\mathbb{C}}
%\renewcommand{\qedsymbol}{\filledbox}

%------------ENVIRONMENTS------------ 
\newenvironment{theorem}[2][]{\begin{trivlist}
\item[{\bfseries #1}\hskip \labelsep {\bfseries #2.}]}{\end{trivlist}}
\newenvironment{lemma}[2][Lemma]{\begin{trivlist}
\item[\hskip \labelsep {\bfseries #1}\hskip \labelsep {\bfseries #2.}]}{\end{trivlist}}
\newenvironment{exercise}[2][Exercise]{\begin{trivlist}
\item[\hskip \labelsep {\bfseries #1}\hskip \labelsep {\bfseries #2.}]}{\end{trivlist}}
\newenvironment{reflection}[2][Reflection]{\begin{trivlist}
\item[\hskip \labelsep {\bfseries #1}\hskip \labelsep {\bfseries #2.}]}{\end{trivlist}}
\newenvironment{proposition}[2][Proposition]{\begin{trivlist}
\item[\hskip \labelsep {\bfseries #1}\hskip \labelsep {\bfseries #2.}]}{\end{trivlist}}
\newenvironment{corollary}[2][Corollary]{\begin{trivlist}
\item[\hskip \labelsep {\bfseries #1}\hskip \labelsep {\bfseries #2.}]}{\end{trivlist}}
\theoremstyle{remark}
\newtheorem*{remark}{Remark}

%------------------------------------ 
%---------START-OF-DOCUMENT----------
%------------------------------------
\begin{document}
 
\title{Homework 2}
\author{David Miller \\ 
MAP 5165: Methods of Applied Mathematics I} 
 
\maketitle

\subsection*{Problem 1}

\textit{Find the fixed points and classify them.}
\begin{align}
	\dot{x} = y - x^3 + x, \quad \dot{y} = -x - y^3 + y
\end{align}

The point (0,0) is an obvious fixed point. Now lets check for other fixed points:
\begin{align*}
	x = 0 = y - x^3 + x & \Rightarrow y = x^3 - x \\
	y = 0 = -x - y^3 + y & \Rightarrow 0 -x - (x^3-x)^3 + x^3 - x \\
	& \Rightarrow 0 = -1 - x^2(x+1)^3(x-1)^3 + (x+1)(x-1)
\end{align*}
From this we can see (from plugging into Wolfram$|$Alpha) that there are no other fixed points. Therefore letting $f = y - x^3 + x$ and $g = -x -y^3 + y$ we get 
\begin{center}
\begin{pmatrix} 
f_x & f_y \\ 
g_x & g_y 
\end{pmatrix} =
\begin{pmatrix}
-3x^2 + 1 & 1 \\
-1 & -3y^2 + 1
\end{pmatrix}
 \xrightarrow{\text{Plugging in }(0,0)}
\begin{pmatrix}
1 & 1 \\ -1 & 1
\end{pmatrix}
\xrightarrow{\text{Compute eigenvalue }\lambda}
\begin{pmatrix}
	1-\lambda & 1 \\ -1 & 1 - \lambda
\end{pmatrix}
\end{center}
\vspace{0.25cm}

From this we have that $(1-\lambda)^2 = -1$ and therefore $\lambda = 1\pm i$. Therefore the fixed point is an unstable spiral.

\newpage

\subsection*{Problem 2}

\textit{A particle is confined on the half line $x \geq 0$ and moves according to the following equation of motion: $\dot{x} = -x^\alpha$ where $\alpha$ is a real constant.} \\ \\
\textit{(a) Find all the values of $\alpha$ for which the origin $(x = 0)$ is a stable fixed point.} \\

If we want the origin to be a stable fixed point then we need two conditions
\begin{itemize}
	\item If $x > 0$ then $\dot{x}$ must be less than zero to force it back to the origin,
	\item If $x < 0$ then $\dot{x}$ must be greater than zero to force  it back to the origin.
\end{itemize}
When $x > 0$ we get that $\alpha$ can be any positive real number. If it is negative than the problem does ot admit a fixed point. When $x < 0$ we get that $\alpha$ must be some rational $\frac{p}{q}$ where $p$ is an odd integer. Putting this together we get that $\alpha$ can take on any positive number on the positive half-line. \\


\textit{(b) Consider the values of $\alpha$ found in part (a). Can the particle ever reach the origin in finite time? Specifically, how long does it take for the particle to travel from $x = 1$ to $x = 0$ as a function of $\alpha$.} \\

\begin{align*}
	\frac{dx}{dt} & = -x^\alpha \\
	-\int\frac{dx}{x^{\alpha}} & = \int dt \\
	-\frac{x^{-\alpha+1}}{-\alpha+1} & = t + c\\
\end{align*}

for some $c \in \mathbb{R}$. The time it takes to get from $x=1$ to $x=0$ is
$$ t_0 - t_1 = \frac{1}{1 - \alpha} $$
Therefore we do have that the particle can reach the origin in finite time if we allow $\alpha \in (0,1)$.
\newpage 

\subsection*{Problem 3}

\textit{Consider the equation}
\begin{align}
	& \dot{x} = cx + x^3
\end{align}

\textit{where $x(t) \in \R$ and $c > 0$ is real and fixed. Prove rigorously that $x(t) \rightarrow \pm\infty$ in finite time, starting from any initial condition $x_0 \neq 0$.} \\
\begin{align*}
\frac{dx}{dt} & = cx + x^3 \\
\int \frac{dx}{x(c + x^2)} & = \int dt \\
\int \frac{dx}{cx} - \int \frac{xdx}{c(c+x^2)} & = t + a \\
\frac{ln(x)}{c} - \frac{ln(c+x^2)}{2c} & = t + a \\
\frac{1}{c}ln\bigg(\frac{x}{c + x^2}\bigg) & = t + a
\end{align*}
for some $a \in \mathbb{R}$. We can not have that $x_0 = 0$ since we can not evaluate $ln(0)$. As $x \rightarrow \pm\infty$ we have that $ln(\frac{x}{c+x^2}) \rightarrow 0$. Therefore there is some finite $t^*$ such that $x(t^*) \rightarrow \pm\infty$.

\newpage

\subsection*{Problem 4}

\textit{Find and classify all equilibrium points and sketch the local phase diagrams (find all phase paths whenever possible):} \\ \\
\textit{(a) $\dot{x} = sin(y), \quad \dot{y} = -sin(x)$} \\ \\
From the problem we have that  
$$ sin(y) = 0 \text{ and} -sin(x) = 0 $$
and therefore the fixed points are $(x,y) = (n\pi,m\pi)$ for $n,m \in \mathbb{Z}$. Setting $f = sin(y)$ and $g = -sin(x)$ we get \\
\begin{center}
\begin{pmatrix} 
	f_x & f_y \\ 
	g_x & g_y 
\end{pmatrix} =
\begin{pmatrix}
	0 & cos(y) \\
	cos(x) & 0
\end{pmatrix}
\xrightarrow{\text{Plugging in ($n\pi,m\pi$)}}
\begin{pmatrix}
	 0 & cos(m\pi) \\ cos(n\pi) & 0
\end{pmatrix}
\xrightarrow{\text{Compute }\lambda}
\begin{pmatrix}
	-\lambda & cos(m\pi) \\ cos(n\pi) & \lambda
\end{pmatrix}
\end{center} \\

\vspace{0.25cm}
Solving this we get $\lambda^2 - cos(n\pi)cos(m\pi) = 0$ where $cos(n\pi)cos(m\pi)$ is always 1 or -1. If $cos(n\pi)cos(m\pi)$ = 1 we have that $\lambda = \pm i$ and therefore we have center fixed point. If $cos(n\pi)cos(m\pi)$ = -1 then we have that $\lambda = \pm 1$. This is a saddle point and therefore can not be a stable fixed point. \\

\newpage

\textit{(b) $\dot{x} = 4 - 4x^2 - y^2, \quad \dot{y} = 3xy$} \\ \\
From the problem we have that 
$$ 3xy = 0 \, \Rightarrow \, x = y = 0 \, \Rightarrow \, 4 - 4x^2 = 0 \text{ and } 4 - y^2 = 0 \, \Rightarrow \, y = \pm 2, \, x = \pm 1$$
So we have that the four fixed points are $(x,y) = (0,\pm2)$ and $(x,y) = (\pm1,0)$. Setting $f = 4 - 4x^2 - y^2$ and $g = 3xy$ we get

\begin{center}
	\begin{pmatrix} 
		f_x & f_y \\ 
		g_x & g_y 
	\end{pmatrix} =
	\begin{pmatrix}
		-8x & -2y \\
		3y & 3x
	\end{pmatrix}
	\xrightarrow{\text{Plugging in (0,$\pm 2$)}}
	\begin{pmatrix}
		0 & \mp4 \\ \pm6 & 0
	\end{pmatrix}
	\xrightarrow{\text{Compute eigenvalue }\lambda}
	\begin{pmatrix}
		-\lambda & \mp4 \\ \pm6 & -\lambda
	\end{pmatrix}
\end{center} \\
\vspace{0.25cm}

Solving this we get $\lambda^2 + 24$ = 0. Therefore we have that $\lambda = \pm i\sqrt{24}$ which is a center for the fixed point $(0,\pm2)$. Plugging in $(\pm1,0)$ yields $(8 \pm \lambda)(3 \pm \lambda)$ which evaluates to $\lambda = \sqrt{24}$. This implies that $(\pm1,0)$ is a saddle point and therefore not a stable fixed point.

\newpage

\subsection*{Problem 5}

\textit{Prove that the ODE}
\begin{align}
	\dot{x} = 1 + x^{12}, \quad x(0) = 2
\end{align}

\textit{blows up in finite time.} \\

We have that  
$$  \dot{x}_1 = x^{12} \, \leq \, \dot{x} = 1 + x^{12}, \quad \forall x $$
Therefore it is sufficient to show $\dot{x}_1$ blows up in finite time.
\begin{align*}
	\frac{dx_1}{dt} & = x_1^{12} \\
	\int\frac{dx_1}{x_1^{12}} & = \int dt \\
	-\frac{1}{11x_1^{11}} & = t + c \\
	\Rightarrow c & = \frac{-2^{-11}}{11} \\
	x_1 & = \frac{-1}{11}\frac{1}{t+c} 
\end{align*}
 Therefore we have that $\dot{x}_1$ blows up in finite time ($t = \frac{1}{11\star 2^{11}}$) and therefore $\dot{x}$ blows up in finite time. 

\newpage

\subsection*{Problem 6}

\textit{Prove the following theorem:}
\begin{theorem}{Theorem}
	\textit{Suppose that $\Gamma$ lies in a simply connected domain in $\R^2$ and there are no fixed points in the area enclosed by $\Gamma$, then the index $I_{\Gamma}$ must be zero.} 
\end{theorem}
\begin{proof}
Since $\Gamma$ is a closed non-intersecting curve in a simply connected region we can apply Green's Theorem which states
$$ \oint_\Gamma (P dx + Q dy) = \iint_{D\Gamma} \bigg(\frac{\partial Q}{\partial x} - \frac{\partial P}{\partial y}\bigg)dxdy $$
where $D_\Gamma$ is the interior region bounded by $\Gamma$. In class we were given the equation for calculating the index
$$ I_\Gamma = \frac{1}{2\pi}\int\limits_{s_0}^s \frac{XY^\prime - YX^\prime}{X^2 + Y^2} \, ds $$
Since we want to evaluate this in $\mathbb{R}^2$ to use Green's Theorem we use the chain rule
$$ X^\prime = \frac{dX}{ds} = \frac{dx}{ds} + X_y\frac{dy}{ds}, \quad Y^\prime = \frac{dY}{ds} = Y_x\frac{dx}{ds} + Y_y\frac{dy}{ds} $$
which then turns the index equation into
\begin{align*}
I_\Gamma & = \frac{1}{2\pi}\oint_\Gamma\bigg(\underbrace{\frac{XY_x - YX_x}{X^2 + Y^2}}_{P}\,dx + \underbrace{\frac{XY_y - YX_y}{X^2 + Y^2}}_{Q}\,dy\bigg).
\end{align*}
The functions $P$ and $Q$ satisfy Green's Theorem since the denominator of each does not equal zero. Therefore we have 
$$ I_\Gamma = \frac{1}{2\pi}\iint_\Gamma\bigg[\bigg(\frac{\partial}{\partial x}\frac{XY_y - YX_y}{X^2 + Y^2}\bigg) - \frac{\partial}{\partial y}\bigg(\frac{XY_x - YX_x}{X^2 + Y^2}\bigg)\bigg] $$
Evaluating the inside of the double integral we get
\begin{align*}
	& (X_xY_y - XY_{yx} - Y_xX_y - YX_{yx})(X^2 + Y^2) - (2XX_x + 2YY_x)(XY_y - YX_y) \\
	- & (X_yY_x + XY_{xy} - Y_yX_x - YX_{xy})(X^2 + Y^2) + (2XX_y + 2YY_y)(XY_x - YX_x)
\end{align*} 
After some very tedious algebra this expression reduces to zero proving that the index of the curve is zero.
\end{proof}

\end{document}